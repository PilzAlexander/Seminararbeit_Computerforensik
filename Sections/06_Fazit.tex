In dieser Seminararbeit wurde eine forensische Analyse einer .pst-Datei durchgeführt. Hierfür wurden verschiedene Analysen durchgeführt. Die Analysen wurde mithilfe von eigens geschriebenen Python-Skripten durchgeführt, da Tools für die Analyse von .pst-Dateien nur kostenpflichtig zur Verfügung standen bzw. bei den kostenfreien Versionen nur sehr eingeschränkte Funktionalitäten besessen haben. Durch die Verwendung professioneller Tools wären mit großer Wahrscheinlichkeit noch viel detailliertere Ergebnisse zustande gekommen. 

Zum Parsing der .pst-Datei wurde die Python Bibliothek "libpff" verwendet, die sich speziell für den Zugriff auf Dateiformate von Microsoft Outlook eignet. Mithilfe dieser Bibliothek wurden dann die wichtigsten Eigenschaften der E-Mails extrahiert, um sie für die weitere Analyse aufzubereiten. Im Anschluss daran wurden dann eine Senderanalyse, eine zeitliche Analyse und eine Spamwortanalyse durchgeführt. 

Die Senderanalyse hat gezeigt, dass Facebook mit 752 E-Mails auf Platz 1 der meisten gesendeten E-Mails liegt. Dies liegt höchstwahrscheinlich daran, dass der Besitzer des Postfaches seine Facebook Benachrichtigungen aktiviert hat. Auf dem zweiten Platz liegt jedoch "Lidl Insider" mit 250 E-Mails. Dieses Ergebnis ist erschreckend, da ich nach einer Rücksprache mit dem Besitzer des E-Mail Kontos erfahren habe, dass genau eine Bestellung vor einigen Jahren durchgeführt wurde und seitdem zahlreiche E-Mails mit Werbung empfangen werden. 

Die Analyse der Empfangszeiten der E-Mails hat gezeigt, dass deutliche Muster von Häufungen der empfangenen E-Mails zu bestimmten Zeiträumen erkennbar sind. In dem verwendeten Datensatz war klar erkennbar, dass sich eine Häufung zwischen 16 und 18 Uhr abzeichnet. Dies ist für viele Personen die Zeit um Feierabend zu machen. Das ist insofern eine gute Zeit, da viele Leute dann erschöpft und somit unvorsichtiger sind und leichter auf eine Spam-Mail hereinfallen. 

Bei der Analyse der auftretenden Spam-Wörter in den E-Mails wurde eine mithilfe häufig verwendeter Wörter eine "Spam-Wort-Liste" erstellt. Die Inhalte der E-Mails wurden dann mit der Liste überprüft um zu sehen wie häufig bestimmte Wörter auftreten. Dabei war "Angebot" mit 5092 Treffern klar auf Platz 1 und "Date" mit 2991 Treffern auf Platz 2. Somit kam das Wort Angebot im Schnitt in 1,2 E-Mails vor. 

Zu erwähnen ist jedoch, dass diese Ergebnisse nicht repräsentativ sind. Um bessere Ergebnisse zu erzielen bräuchte man zum einen mehr E-Mails und zum Anderen mehrere E-Mail Konten verschiedener Benutzer um andere Verhaltensmuster im zu analysierenden Datensatz vorzufinden.