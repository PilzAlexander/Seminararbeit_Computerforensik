Eine Analyse der Sender wurde durchgeführt, um festzustellen, wie viele verschiedene Absender einer E-Mail an das Analysierte Postfach gesendet haben und welche davon am häufigsten vorhanden waren. Bei dieser Analyse wurde die Eigenschaft \glqq{}sender\grqq{} der E-Mails betrachtet. Hierfür wurde eine Funktion \glqq{}get\_sender\_from\_csv\_file\grqq{} definiert. Diese Funktion durchläuft die .csv-Datei mit einer for-Schleife wobei geprüft wird, ob die Mail-Adresse der entsprechenden Zeile bereits in den Daten vorhanden war oder nicht und der Zähler dieser Zeile erhöht. Falls ein Eintrag noch nicht vorhanden war, wird ein neuer Eintrag in der Liste erstellt und weiter gesucht. Der restliche Code in diesem Skript dient dazu, eine .csv-Datei zu erstellen mit den Spalten \glqq{}Sender\grqq{} und \glqq{}Received Mails\grqq{} und die gezählten Emails in diese Datei einzutragen. Der Code für das Python-Skript ist in Abbildung \ref{fig:countemailssender} abgebildet. Der daraus entstandene Graph wurde auch hier wieder per Import der .csv-Datei in eine Excel Arbeitsmappe erstellt. Auf der Abbildung \ref{fig:receivedemails} wurden nur die Absender mit mehr als 30 Treffern abgebildet, um die Übersicht zu behalten. Erkennbar ist Facebook mit 752 Treffern und Lidl Insider mit 250 Treffern. Als Dritter in der Liste war Web.de mit 192 Treffern sehr auffällig. Dies liegt daran, dass ein FreeMail Postfach verwendet wird und deshalb ständig \glqq{}Info-Mails\grqq{} vom Mail-Anbieter im Postfach auftauchen. Insgesamt waren in dem E-Mail Postfach 714 verschiedene Absender vorhanden. Dabei ist mir aufgefallen, dass auch vermehrt gleiche Absender mit Variationen ihrer Senderinformationen vorhanden waren und somit nicht als gleicher Absender erkannt wurden. Erstaunlich ist, dass Lidl Insider mit 250 Treffern vorkommt, obwohl über das Postfach nur eine Bestellung bei Lidl getätigt wurde. Facebook tritt so oft auf, da der Nutzer des Postfachs seine E-Mail Benachrichtigungen aktiviert hat.  

\begin{figure}
    \centering
    \includegraphics[width=0.75\textwidth]{images/Count_Received_Mails_Count_Sender.PNG}
    \caption{Python Code - Zählen der erhaltenen E-Mails und auflisten nach Absender} 
    \label{fig:countemailssender}
\end{figure}

\begin{figure}
    \centering
    \includegraphics[width=0.75\textwidth]{images/Auswertung_Empfange_Emails.png}
    \caption{Absender mit mehr als 100 gesendeten Mails} 
    \label{fig:receivedemails}
\end{figure}