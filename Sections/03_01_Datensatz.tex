Zuerst wird näher auf den verwendeten Datensatz für die Analyse eingegangen. Dabei war die Auswahl eines geeigneten Datensatzes schwieriger als anfangs erwartet, da der Datensatz auch Kriterien für die Analyse erfüllen sollte. Der Datensatz sollte eine relativ hohe Anzahl an E-Mails enthalten, es sollte ein schwacher bzw. kein Spam-Filter vorhanden sein und der Datensatz sollte in ein .pst-Dateiformat überführt werden können. Eine Online-Suche nach geeigneten Datensätzen ergab eine Anzahl an Treffern, die E-Mail-Postfächer simulieren, diese eigneten sich aufgrund ihrer Formate jedoch nicht für die Weiterverarbeitung die in dieser Seminararbeit geplant war.

Aufgrund dessen wurde im privaten Umfeld nach verfügbaren Postfächern gesucht, die ich für die Analyse verwenden durfte. Dabei wurde mir ein E-Mail Konto vom Anbieter "Web.de" zur Verfügung gestellt, welches 4102 E-Mails enthält. Dieses Konto wurde dann in Microsoft Outlook eingebunden um mithilfe des Mail-Programmes einen .pst-Export erstellen zu lassen, der dann als Grundlage für die Analyse verwendet wurde.