\noindent Zuerst wird näher auf den verwendeten Datensatz für die Analyse eingegangen. Dabei war die Auswahl eines geeigneten Datensatzes schwieriger als anfangs erwartet, da der Datensatz auch bestimmte Kriterien für die Analyse erfüllen sollte. Der Datensatz sollte eine relativ hohe Anzahl an E-Mails enthalten, es sollte ein schwacher bzw. kein Spam-Filter vorhanden sein und der Datensatz sollte in ein .pst-Dateiformat überführt werden können, wobei letzteres auch mithilfe von Konvertierungen aus anderen Postfach-Formaten erreichbar gewesen wäre. \smallskip

\noindent Eine Online-Suche nach geeigneten Datensätzen ergab eine Anzahl an Treffern, die E-Mail-Postfächer simulieren, diese eigneten sich aufgrund ihrer Formate jedoch nicht für die Weiterverarbeitung, wie sie die in dieser Seminararbeit geplant war. Aufgrund dessen wurde im privaten Umfeld nach verfügbaren Postfächern gesucht, die ich für die Analyse verwenden durfte. Dabei wurde mir ein E-Mail Konto vom Anbieter \glqq{}Web.de\grqq{} zur Verfügung gestellt, welches 4102 E-Mails enthält. Dieses Konto wurde dann in Microsoft Outlook eingebunden, um mithilfe des Mail-Programmes einen .pst-Export erstellen zu lassen, welcher dann als Grundlage für die Analyse verwendet wurde. Der mir hier zur Verfügung gestellte Datensatz enthält voraussichtlich eine sehr gute Mischung verschiedenster empfangener Mails, was sich sehr gut für eine Analyse eignet. \smallskip

\noindent Zu Vergleichszwecken wurde ein zweiter Datensatz erstellt, der sich aus drei Datensätzen zusammensetzt. Dazu wurde mithilfe von Microsoft Outlook zuerst ein leeres .pst-File erstellt. Dann wurden die drei weiteren Postfächer in Outlook eingebunden. Diese wurden nach ihrer Synchronisation als .pst-Datei exportiert. Anschließend wurden sie durch die Import-Funktion von Outlook in das leere .pst-File importiert. Somit entstand ein neuer Datensatz, der mit dem ersten .pst-File hinsichtlich der Analyseergebnisse verglichen werden kann. Der zweite Satz an Daten enthält insgesamt 6692 E-Mails und bewegt sich daher in einem ähnlichen Größenumfeld wie der erste Datensatz, was die Analyseergebnisse besser vergleichbar macht. Anzumerken ist, dass es sich darunter leider 5653 E-Mails von \glqq{}Twitch\grqq{} befinden, was die Diversität des Datensatzes sehr stark einschränkt. Weil mir keine anderen Postfächer mehr zur Verfügung standen, wurde der Datensatz trotzdem verwendet, um weitere Daten zu analysieren.