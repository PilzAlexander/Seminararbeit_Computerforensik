Sobald der Ransomware-Client erfolgreich im Fahrzeug installiert wurde, wird die eigentliche Geiselnahme der Ransomware durchgeführt. Unter Geisel ist hierbei eine gesperrte Komponente, die nicht leicht wiederhergestellt werden kann oder bei der eine lange Ausfalldauer nicht vertretbar ist, die Beschlagnahme oder angedrohte Veröffentlichung von internen Daten, durch die ein erheblicher Schaden verursacht wird oder etwas anderes zu verstehen, um das Opfer zur Zahlung des Lösegeldes zu zwingen. Sobald ein Angreifer Zugriff auf das System hat, kann er nach Belieben vorgehen, um das Opfer zu schädigen und somit eine Lösegeldzahlung zu erreichen. Beispiele für Angriffe sind unter anderem das Blockieren einer Türverriegelung, Sperren wichtiger kryptografischer Zugangsdaten wie dem Fahrzeugschlüssel, die Verschlüsselung kritischer fahrzeuginterner Daten, Einschüchterung des Opfers mithilfe gefälschter kritischer technischer Zustände wie einer überhitzten Batterie, oder reale physische Manipulationen wie dem Auslösen aller Airbags oder Abschalten des Motors. Die Möglichkeiten sind praktisch unbegrenzt.

Nach erfolgter Geiselnahme wird dies auch für das Opfer sichtbar. Dabei wird klar und deutlich erkennbar, was passiert ist und was die notwendigen Schritte sind, um wieder Zugriff auf sein System zu erhalten. Zur Veranschaulichung wie es aussehen könnte, wenn ein System übernommen wurde, haben die Autoren des dieser Arbeit zugrundeliegenden Artikels einen Demonstrator erstellt, der auf Abbildung  zu sehen ist. 

%\begin{figure}
%    \centering
%    \includegraphics[width=0.75\textwidth]{images/WannaDrive.png}
%    \caption{WannaDrive? Ransomware Demonstration 
%    \cite{M.Wolf.2017}}
%    \label{fig:wannadrive}
%\end{figure}