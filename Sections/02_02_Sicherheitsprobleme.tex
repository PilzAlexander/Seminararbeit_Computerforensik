\noindent pPST-Dateien bieten den Vorteil, dass sie leicht übertragbar sind. Auch ohne fundierte IT-Kenntnisse können sie zwischen verschiedenen Outlook-Clients übertragen werden. Abgesehen von diesem Mobilitätsvorteil bringen sie jedoch auch Sicherheitsprobleme mit sich \cite{.26.06.2022}:

\begin{itemize}
    \item Beschädigungs- bzw. fehleranfällig, was zu Datenverlusten führen kann
    \item Stromausfälle, PC-Abstürze oder versehentliches Schließen trennen eine PST-Datei vom Outlook-
    Profil (abgekoppelte oder verwaiste PSTs entstehen, die für die IT-Abteilung nicht sichtbar sind,
    auch wenn sie wertvolle Informationen enthalten, die benötigt werden)
    \item Inhalte von PST-Dateien sind nur an der Quelle verfügbar und somit bei Analysen von zentralen Punkten aus eventuell nicht verfügbar
    \item Zugangsbeschränkungen oder eine falsche Klassifizierung können dazu führen, dass PST-Objekte
    falsch kategorisiert werden
    \item Passwortschutz vorhanden, kann aber sehr leicht entschlüsselt werden
    \item Native PST-Verschlüsselung bietet keinen ausreichenden Schutz
\end{itemize}