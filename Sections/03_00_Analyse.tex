\noindent In diesem Kapitel werden die durchgeführten Analyseschritte näher erläutert. Zuerst wird auf das Finden geeigneter Datensätze näher eingegangen. Anschließend wird das Parsen der .pst-Datei dargestellt. Danach wird auf eine Senderanalyse, eine zeitliche Analyse sowie eine Analyse bestimmter Schlagwörter eingegangen. Zu Beginn der Seminararbeit fand eine Online-Recherche statt, da ich mit dem Thema der forensischen Analyse von .pst-Dateien nicht vertraut war. Dabei bin ich auf Tools zur Analyse von PST-Files gestoßen und wollte mithilfe dieser die Analyse vornehmen. Jedoch erwiesen sich die Gratis-Versionen dieser Tools als unbrauchbar für meine Zwecke, da sie mir keine tiefergehenden Einblicke in die Datei ermöglichten, sondern lediglich die E-Mails betrachten ließen wie bei Outlook. Ein Kauf von Lizenzen kam für mich dabei nicht in Frage, da diese meist sehr teuer sind. Nach diesem Ausschluss der fertigen Tools stellte sich die Frage, wie man denn diese Dateien noch analysieren könne. \smallskip


\noindent Nach weiterer Recherchearbeit bin ich auf eine Bibliothek der Programmiersprache Python gestoßen, die genau für diese Analysezwecke entwickelt wurde. Nach dem Betrachten der Dokumentation und einiger Code-Beispiele entschloss ich mich dazu, die Analyse mithilfe selbstgeschriebener Skripte durchzuführen. Python eignet sich aus meiner Sicht besonders gut für diesen Einsatzzweck, da die Programmiersprache eine breite Auswahl an Bibliotheken und Modulen zur Verfügung hat, mit denen man Daten analysieren, auswerten und darstellen lassen kann. Durch die dynamische Typisierung eignet sich Python außerdem besonders für Skripte und eine schnelle Entwicklung von Anwendungen. Für die Durchführung der Analyse wurden folgende Schritte geplant:

\begin{enumerate}
    \item geeigneten Datensatz finden
    \item .PST-Datei erzeugen 
    \item Parsen der .PST-Datei in eine CSV-Datei
    \item Senderanalyse
    \item Zeitliche Analyse
    \item Analyse von Spam-Wörtern
\end{enumerate}