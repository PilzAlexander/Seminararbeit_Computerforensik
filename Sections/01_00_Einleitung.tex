E-Mails sind im privaten wie auch im geschäftlichen Umfeld nicht mehr als Kommunikationsmedium wegzudenken. Dies ist klar erkennbar an den versandten und empfangenen E-Mails weltweit. So lag diese Zahl im Jahr 2021 bei 319,6 Milliarden E-Mails und steigt laut einer Prognose bis ins Jahr 2025 auf 376,4 Milliarden E-Mails \cite{Statista.26.06.2022}. 

E-Mails werden heutzutage aber nicht nur für die Kommunikation verwendet, sondern auch zum Versenden von Spamnachrichten. So beläuft sich der Anteil an Spam-Mails am weltweiten Anteil der versandten E-Mails auf 46 Prozent \cite{ASITZentrumfursichereInformationstechnologieAustria.26.06.2022}. Unter Spam versteht man Nachrichten, die ohne Aufforderung und unterwünscht zugestellt werden. Meist haben diese Nachrichten den Zweck Werbung zu verbreiten. Jedoch gibt es auch weitaus bedenklichere Nachrichten. So wird mithilfe von Spam-Mails versucht, unvorsichtige Nutzer dazu zu bringen, persönliche Daten preiszugeben oder finanzielle Gewinne zu erzielen. 

Im geschäftlichen Umfeld wird hier dem Nutzer meist die Arbeit abgenommen, weil professionelle Spam-Filter eingerichtet sind und so fast keine unerwünschten Nachrichten mehr ankommen. Im privaten Umfeld werden solche Spamfilter oft nicht bzw. nicht richtig angewandt. Hinzu kommt, dass sich viele Nutzer mit dieser Thematik gar nicht auseinandersetzen. Dies führt häufig dazu, dass die Postfächer der Nutzer hier regelrecht durch Spam-Mails überladen werden. 

Im Rahmen dieser Seminararbeit sollen diese unerwünschten E-Mails mithilfe des .PST-Dateiformats analysiert und die Ergebnisse dokumentiert werden.