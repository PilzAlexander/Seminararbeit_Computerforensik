Das .pst-Dateiformat ist ein proprietäres Dateiformat, welches von Microsoft im Mailprogramm \glqq{}Outlook\grqq{} verwendet wird. PST steht dabei für Personal Storage Table. Microsoft nutzt das Dateiformat zum Speichern von Nachrichtenkopien, Kalendereinträgen und Kontakten. Nutzt man den Microsoft Exchange Server werden die Daten an den Server übermittelt und dort gespeichert. Im Gegensatz dazu speichert Microsoft Outlook ohne Exchange Server diese Elemente auf dem lokalen Computer. Dabei werden .pst-Dateien meist zum Speichern archivierter Elemente verwendet. Zum besseren Verständnis kann eine .pst-Datei auch als ein \glqq{}Aktenschrank\grqq{} bzw. eine \glqq{}Büroablage\grqq{} bezeichnet werden. Diese besitzt beschriftete Schubladen, in denen die verschiedenen Daten kategorisch sortiert abgelegt werden. Hängen diese übergreifend miteinander zusammen, besteht die Möglichkeit Querverweise zwischen den einzelnen Schubladen herzustellen. Der genaue Aufbau dieser Architektur wird im nächsten Abschnitt dieser Arbeit näher erläutert. Die PST-Dateien sind gewöhnlich nur mit Outlook zu öffnen, jedoch gibt es mittlerweile auch andere Tools um diese zu öffnen, bzw. zu analysieren. Auf diese Tools wird an dieser Stelle nicht weiter eingegangen, da ich diese nicht verwendet habe. 

