In diesem Kapitel wird auf die Analyse bestimmter Schlagwörter in den E-Mails eingegangen. Um diese Analyse durchzuführen habe ich im Internet nach gängigen Wörtern gesucht, die auf eine Spam-Mail hinweisen \cite{Heise.07.06.2021}. Anhand der gefundenen Wörter habe ich passende ausgesucht und als Liste in einem Python Skript hinterlegt. Diese Liste ist in Abbildung \ref{fig:spamwortliste} zu sehen. Dabei wurden Kategorien wie Finanzen, Glücksspiel, Gesundheit, Shopping, Dating und Sonstige betrachtet. Für die Analyse der Wörter wurde die .csv-Datei verwendet, welche beim Parsen der .pst-Datei entstanden ist. Diese wurde dann mit je einem Wort aus der Spamwort-Liste durchlaufen und ein Counter erhöht, sobald ein Treffer erzielt wurde. Anschließend wurde die Liste \glqq{}word\_list\grqq{} mit den Headern \glqq{}SpamWords\grqq{} und \glqq{}Hits\grqq{} in eine .csv-Datei exportiert um eine grafische Auswertung dieser Daten vorzunehmen. Der hierfür verwendete Python Code ist in Abbildung \ref{fig:spamwordsearch} zu sehen. Die hierdurch erstellte .csv-Datei habe ich im Anschluss in Microsoft Excel importiert und dort eine Graphische Auswertung durchgeführt. Excel habe ich gewählt, da ich anders als bei der Auswertung der zeitlichen Eigenschaften der E-Mails eine andere Art der grafischen Auswertung abseits von Python ausprobieren wollte. Das Ergebnis der Auswertung ist in Abbildung \ref{fig:spamwortlistegreater100} zu sehen. Am häufigsten kamen davon das Wort \glqq{}Angebot\grqq{} mit 5092 Treffern, danach \glqq{}Date\grqq{} mit 2991 Treffern gefolgt von \glqq{}klicken\grqq{} mit 1319 Treffern vor. Diese Wörter sind in die Kategorien Shopping und Dating sowie Sonstige einzuordnen. In der Abbildung \ref{fig:spamwortlistegreater100} sind alle Wörter mit einer Trefferanzahl von mehr als 100 abgebildet.

\begin{figure}
    \centering
    \includegraphics[width=0.75\textwidth]{images/Spamwortliste.PNG}
    \caption{Spamwortliste} 
    \label{fig:spamwortliste}
\end{figure}

\begin{figure}
    \centering
    \includegraphics[width=0.75\textwidth]{images/python_Spamwordsearch.PNG}
    \caption{Python Code - Durchsuchen der E-Mails nach Schlagworten} 
    \label{fig:spamwordsearch}
\end{figure}

\begin{figure}
    \centering
    \includegraphics[width=0.75\textwidth]{images/Auswertung_Spamwortsuche.png}
    \caption{Wörter mit mehr als 100 Treffern} 
    \label{fig:spamwortlistegreater100}
\end{figure}